\documentclass[12pt]{article}

%%\usepackage[margin = 1in]{geometry}

\usepackage{graphicx}              
\usepackage{fullpage}              
\usepackage{amsmath}               
\usepackage{amsfonts}              
\usepackage{amsthm}                
\usepackage{amssymb}
\usepackage{mathrsfs}
\usepackage{url}
\usepackage{hyperref}
\usepackage{algorithmic}
\usepackage{algorithm}
\usepackage{authblk}


\hypersetup{
	unicode=true,
	colorlinks=true,
	citecolor=black,
	filecolor=black,
	linkcolor=black,
	urlcolor=black,
	pdfstartview={FitH},
}

% theorem environments
\theoremstyle{plain}
\newtheorem{theorem}{Theorem}
\newtheorem{lemma}[theorem]{Lemma}
\newtheorem{corollary}[theorem]{Corollary}
\newtheorem{proposition}[theorem]{Proposition}
\theoremstyle{definition}
\newtheorem{definition}[theorem]{Definition}
\newtheorem{conjecture}[theorem]{Conjecture}
\newtheorem{example}[theorem]{Example}
\newtheorem*{remark}{Remark}
\newtheorem{note}[theorem]{Note}


\renewcommand{\algorithmicrequire}{\textbf{Input:}}
\renewcommand{\algorithmicensure}{\textbf{Output:}}
\algsetup{linenodelimiter=.}





\newcommand{\wrt}{\vdash} 
\newcommand{\ang}[1]{\{#1\}}
\newcommand{\abs}[1]{\left\vert#1\right\vert}
\newcommand{\dual}[1]{\overline{#1}}
\newcommand{\mapsfrom}{\ensuremath{\reflectbox{$\mapsto$}}}
\newcommand{\tildO}{\tilde{O}}
% roman numerals
\newcommand{\romnum}[1]{\romannumeral #1}
\newcommand{\Romnum}[1]{\uppercase\expandafter{\romannumeral #1}}

\newcommand{\todo}[1]{\textcolor{red}{TODO: #1}}
\newcommand{\comment}[2][Note]{\textcolor{green}{(#1): #2}}

\DeclareMathOperator{\fieldchar}{char} % characteristic of a field
\DeclareMathOperator{\End}{{\rm End}} % endomorphism ring
\DeclareMathOperator{\trace}{Tr} % finite field trace
\DeclareMathOperator{\gal}{{\rm Gal}} % Galois group
\DeclareMathOperator{\order}{ord} % order of an element
\DeclareMathOperator{\lcm}{lcm} % least common multiple
\DeclareMathOperator{\divisor}{div} % divisor on a curve
\DeclareMathOperator{\supp}{supp} % support of a divisor
\DeclareMathOperator{\norm}{N} % norm
\DeclareMathOperator{\Res}{Res}
\DeclareMathOperator{\Aut}{Aut}
\DeclareMathOperator{\minpoly}{minpoly}
\DeclareMathOperator{\loglog}{loglog}
\DeclareMathOperator{\rev}{rev}

\def\Q{\ensuremath{\mathbb{Q}}}
\def\N{\ensuremath{\mathbb{N}}}
\def\R{\ensuremath{\mathbb{R}}}
\def\Z{\ensuremath{\mathbb{Z}}}
\def\F{\ensuremath{\mathbb{F}}}
% \def\H{\ensuremath{\mathbb{H}}}
\def\K{\ensuremath{\mathbb{K}}}
% \def\L{\ensuremath{\mathbb{L}}}
% \def\A{\ensuremath{\mathbb{A}}}
% \def\B{\ensuremath{\mathbb{B}}}
\def\MM{\ensuremath{\mathsf{M}}}
\def\MMM{\ensuremath{\mathsf{MM}}}
% \def\MC{\ensuremath{\mathsf{C}}}
% \def\PC{\ensuremath{\mathsf{PC}}}
% \def\II{\ensuremath{\mathsf{I}}}
% \def\QQ{\ensuremath{\mathsf{Q}}}
\def\CC{\ensuremath{\mathsf{C}}}
% \def\RR{\ensuremath{\mathsf{R}}}
% \def\AA{\ensuremath{\mathsf{A}}}
% \def\va{\ensuremath{\mathsf{a}}}
% \def\vy{\ensuremath{\mathsf{y}}}
% \def\vu{\ensuremath{\mathsf{u}}}
% \def\vb{\ensuremath{\mathsf{b}}}
% \def\vc{\ensuremath{\mathsf{c}}}
% \def\mul{\ensuremath{\mathsf{mul}}}
% \def\rem{\ensuremath{\mathsf{rem}}}
% \def\cat{\ensuremath{\mathsf{cat}}}
% \def\coeff{\ensuremath{\mathsf{coefficient}}}
% \def\mulmod{\ensuremath{\mathsf{mulmod}}}
% \def\x{\ensuremath{\mathbf{x}}}
% \def\uu{\ensuremath{\mathbf{U}}}
% \def\bb{\ensuremath{\mathbf{B}}}
% \def\bxi{\boldsymbol{\xi}}
% \def\bupsilon{\boldsymbol{\upsilon}}
% \def\bzeta{\boldsymbol{\zeta}}
% \def\blambda{\boldsymbol{\lambda}}
\def\euler{\ensuremath{\varphi}}

\newcommand{\D}{\Delta}
\usepackage{tikz}
\usetikzlibrary{matrix,arrows}

\newcommand{\ph}{(\phi/f)}
\newcommand{\m}{\mathfrak m}
\newcommand{\g}{\mathfrak g}
\newcommand{\be}{\mathfrak b}

\newcommand{\p}{\mathfrak p}
\newcommand{\h}{\mathfrak h}
\newcommand{\f}{\mathfrak f}
\newcommand{\q}{\mathfrak q}
\newcommand{\el}{\mathfrak l}
\newcommand{\B}{\mathfrak B}
\newcommand{\ef}{\mathcal F}


\setlength{\parindent}{0mm}





\title{Drinfeld Modules with Complex Multiplication, Hasse Invariants and Factoring Polynomials over Finite Fields}



\author[1]{Javad Doliskani}
\author[2]{Anand Kumar Narayanan}
\author[3]{\'Eric Schost}

\affil[1]{\small Institute for Quantum Computing, University of Waterloo}
\affil[2]{\small Computing and Mathematical Sciences, Caltech}
\affil[3]{\small Computer Science Department, University of Waterloo}

\date{}



\begin{document}
\maketitle

\begin{abstract}
\noindent 

We present a novel randomized algorithm to factor polynomials over a
finite field $\F_q$ of odd characteristic using rank $2$ Drinfeld
modules with complex multiplication. The main idea is to compute a
lift of the Hasse invariant (modulo the polynomial $f(x) \in \F_q[x]$
to be factored) with respect to a random Drinfeld module $\phi$ with
complex multiplication. Factors of $f(x)$ supported on prime ideals
with supersingular reduction at $\phi$ have vanishing Hasse invariant
and can be separated from the rest.



Incorporating a Drinfeld module analogue of Deligne's congruence, we
devise an intricate method to compute the Hasse invariant lift, which
turns out to be the crux of our algorithm. The resulting expected
runtime of $n^{3/2+o(1)} (\log q)^{1+o(1)}+n^{1+o(1)} (\log
q)^{2+o(1)}$ to factor polynomials of degree $n$ over $\F_q$ matches
the fastest previously known algorithm, the Kedlaya-Umans
implementation of the Kaltofen-Shoup algorithm. In comparison, our
algorithm has the distinction of not requiring separate distinct
degree and equal degree factorization phases, potentially leading to a
faster implementation. We augment our theoretical results with a
subquadratic implementation of our algorithm that from preliminary
experimentation appears to the fastest way to factor polynomials over
finite fields.

% A Drinfeld module analogue of Deligne's congruence plays a key role in computing the Hasse 
%invariant lift. We devise an intricate algorithm to compute the Hasse invariant lift yielding an 
%expected runtime of $n^{3/2+o(1)} (\log q)^{1+o(1)}+n^{1+o(1)} (\log q)^{2+o(1)}$ to factor 
%polynomials of degree $n$ over $\F_q$. This matches the fastest previously known algorithm, the 
%Kedlaya-Umans implementation of the Kaltofen-Shoup algorithm. Unlike the Kaltofen\\ \\
 
%We present two algorithms based on this idea. The first algorithm  chooses Drinfeld modules with complex multiplication at random and has a quadratic expected run time. The second is a deterministic algorithm with $O(\sqrt{p})$ run time dependence on the characteristic $p$ of $\F_q$.
\end{abstract}



\section{Introduction}

Drinfeld modules of rank two are often presented as an analogue over
function fields such as $\F_q(x)$ (for a prime power $q$) of elliptic
curves over number fields. In very concrete terms, a Drinfeld module
over $\F_q(x)$ is simply a ring homomorphism $\phi$ (together with
some mild assumptions) from $\F_q[x]$ to the ring of skew polynomials
$\F_q(x)\ang{\tau}$, where $\tau$ satisfies the commutation
relation $\tau u = u^q \tau$ for $u$ in $\F_q(x)$. The rank of a
Drinfeld module is the degree of $\phi(x)$.

In this definition, one may replace $\F_q(x)$ by any other other field
$L$ equipped with a homomorphism $\F_q[x]\to L$, and in particular by
a finite field of the form $L=\F_q[x]/f$; one may then define the {\rm
  reduction} of a Drinfeld module over $\F_q(x)$ modulo an irreducible
$f \in \F_q[x]$. Then, there exist striking similarities between the
theory of elliptic curves over $\Q$ and their reductions modulo primes,
and that of rank two Drinfeld modules over $\F_q(x)$ and their
reduction modulo irreducibles $f$. For instance, such notions as
endomorphism ring, complex multiplication, Hasse invariants,
supersingularity, or the characteristic polynomial of the Frobenius,
\dots~can be defined in both contexts, and share many
properties.

While the literature on algorithmic aspects of elliptic curves is
extremely rich, this is not the case for Drinfeld modules; only
recently have they been considered under the algorithmic viewpoint.  A
first goal in this paper is to give algorithms for the computation of
the Hasse invariant of a rank two Drinfeld module over a finite field.

The Hasse invariant $h_E$ of an elliptic curve $E: y^2=f(x)$ over a
finite field $\F_p$, with $f$ of degree $3$, can be defined as the
coefficient of degree $p-1$ in $f^{(p-1)/2}$ (other definitions set it
to be $1$ if this coefficient is nonzero, $0$ otherwise). The
definition of $h_E$ makes it possible to compute it using a number of
operations softly linear in $p$, but one can do better: it is possible
to compute $h_E$ without computing all previous coefficients, using
the fact that the coefficients of the $f^{(p-1)/2}$ satisfy a linear
recurrence with polynomial coefficients, and applying techniques for
such recurrences due to Strassen~\cite{Strassen76} and the
Chudnovskys~\cite{ChCh88}.

In the case of a rank-two Drinfeld module defined by a ring morphism
$\phi: \F_q[x] \to L\ang{\tau}$, with $L=\F_q[x]/f$ and $f$ of degree
$k$, the Hasse invariant $h_\phi$ is now defined as the coefficient of
$\tau^k$ in $\phi(f) =\sum_{i=0}^{2k} h_i \tau^i$, with $h_i$ in $L$
for all $i$ (all coefficients of index less than $k$ vanish).  As in
the elliptic case, one can immediately deduce an algorithm from the
definition; our first contribution is to show that better algorithms
exist, based on a recurrence (due to Gekeler~\cite{gek}) somewhat 
similar to the one used in the elliptic case.



Let $q$ be a power of a an odd prime $p$ and let $\F_q$ denote the
finite field with $q$ elements.  The univariate polynomial
factorization problem over $\F_q$ is the following:
\begin{itemize}
 \item \textsc{Polynomial Factorization:} \textit{Given a monic $f(x) \in \mathbb{F}_q[x]$ of 
 degree $n$, write $f(x)$ as a product of its monic irreducible factors.}
\end{itemize}


Without loss of generality, we may assume that $f(x)$ is squarefree
\cite{knu,yun}, that is, does not contain a square of an irreducible
polynomial as a factor. Berlekamp showed that \textsc{Polynomial
  Factorization} can be solved in randomized polynomial time
\cite{ber} and there is an extensive line of research
\cite{cz,gs} with the best known bound being the improvement by
Kedlaya and Umans~\cite{ku} of an algorithm due to Kaltofen and
Shoup~\cite{ks}.


The use of Drinfeld modules to factor polynomials over finite fields
originated with Panchishkin and Potemine \cite{pp}, whose algorithm
was rediscovered by van der Heiden \cite{vdH}. These algorithms, along
with the second author's Drinfeld module black box Berlekamp
algorithm~\cite{nar} are in spirit Drinfeld module analogues of
Lenstra's elliptic curve method to factor integers \cite{len}. The
Drinfeld module degree estimation algorithm of \cite{nar} uses
Euler-Poincar\'e charactersitics of Drinfeld modules to estimate the
factor degrees in distinct degree factorization. A feature common to
the aforementioned algorithms is their use of random Drinfeld
modules, which typically don't have complex multiplication. 


In this paper we take a different approach. We construct random rank
$2$ Drinfeld modules $\phi$ with complex multiplication by an
imaginary quadratic extension of the rational function field $\F_q(x)$
with class number $1$. At roughly half of the prime ideals $\f$ in
$\F_q[x]$, $\phi$ has supersingular reduction. The Hasse invariant of
$\phi$ at a prime ideal $\f$ vanishes if and only if $\phi$ has
supersingular reduction at $\f$. A Drinfeld module analogue of
Deligne's congruence, due to Gekeler \cite{gek}, allows us to compute
a certain lift of Hasse invariants modulo the polynomial $f(x)$ we are
attempting to factor. As a consequence, this lift vanishes exactly
modulo the irreducible factors of $f(x)$ that correspond to the primes
with supersingular reduction. In summary, we get to separate the
irreducible factors corresponding to primes with supersingular
reductions from those with ordinary reduction.


\paragraph{Complexity model.}
We will consider two different complexity models in our runtime
analysis: an \textit{algebraic model} and a \textit{boolean model}. In
the former we count the number of operations $+, \times, \div$ in the
field $\F_q$, while in the latter we count the number of bit
operations, over a standard RAM. 

The main reason behind this dichotomy is that some operations at the
core of our algorithms turn out to be faster in the boolean model.


% The complexity upper bounds for
% polynomial multiplication and modular composition are denoted by
% $\MM(n), \CC(n)$ respectively. In the algebraic model we have $\MM(n)
% \in O(n\log(n)\log\log(n))$ \cite{Schonhage1971}, and $\CC(n) \in
% O(n^{(\omega+1)/2})$ \cite{BrKu78}, where $\omega$ comes from the cost
% $O(n^\omega)$ of square matrix multiplication. We can always take
% $\omega \le 2.37$ \cite{CoWi90} which gives $\CC(n) \in O(n^{1.69})$.

In the boolean model, the modular composition can be done in $O(n^{1 +
  \varepsilon}\log q^{1 + o(1)})$ bit operations \cite{ku}. In this
case, the best polynomial factorization algorithm has the expected run
time of $n^{3/2+o(1)} (\log q)^{1+o(1)}+n^{1+o(1)} (\log q)^{2+o(1)}$
bit operations.  As to if the exponent $3/2$ in $n$ can be lowered
remains an outstanding open question.


\paragraph{Main result.}
The crux of our algorithm is the computation of the Hasse invariant lift modulo $f(x)$. We exploit 
a recursive formula for the lift that arises in Gekeler's proof of a Drinfeld module variant of 
Deligne's congruence \cite{gek}. Our strategy is to first phrase the recurrence in matrix form with 
entries being polynomials modulo $f(x)$. We then observe that solving the recurrence amounts to 
computing the product of a carefully constructed sequence of matrices twited by the Frobenius 
action. The final step is to construct a polynomial with matrix coefficients, whose evaluations 
allow us to rapidly compute the aformentioned product. The evaluations are computed using a fast 
multipoint evaluation algorithm. Our main result can be summarized as:

\begin{theorem}
	\label{theorem:main}
	Given a polynomial $f(x)$ of degree $n$ over $\F_q$, there exits a probabilistic algorithm for 
	factoring $f$ using
	\begin{itemize}
		\item $O(n^{3/2+o(1)} (\log q)^{1+o(1)} + n^{1+o(1)} (\log q)^{2+o(1)})$ bit operations, or
		\item $O(n^{1.815}\log q)$ operations in $\F_q$.
	\end{itemize}
\end{theorem}


The paper is organized as follows. In \S~\ref{drinfeld_section}, Drinfeld modules are introduced 
and the general algorithmic strategy is outlined with emphasis on the role played by Hasse 
invariants and Deligne's congruence. In \S~\ref{randomized_section}, a high level description of 
our algorithm and its rigorous analysis using function field arithmetic is presented. Fast 
computation of the Hasse invariant lift, analysis of the complexity of the algorithm and its 
practical implications are in \S~\ref{sec:hasse}. 
 
 
 
 
%####################################################

\section{Drinfeld modules}

Let $A = \F_q[x]$ be the ring of univariate polynomials over the finite field $\F_q$ where $q$ is a 
prime power. Let $L/\F_q$ be a field with an $\F_q$-algebra homomorphism $\gamma: A \rightarrow 
L$. Let $\tau: \F_q \rightarrow \F_q$ be the Frobenius automorphism and let $L\{\tau\}$ be the 
\textit{skew-polynomial} ring defined by the action $\tau f = f^q\tau$ for all $f \in L$. Given an 
integer $r > 0$, a Drinfeld module of rank $r$ over $L$ is a morphism
\begin{equation}
\label{equ:Drinfeld}
	\begin{array}{rrll}
		\phi : & A & \longrightarrow & L\{\tau\} \\
		& x & \longmapsto & a_0 + a_1\tau + \cdots + a_r\tau^r	
	\end{array}
\end{equation}
with $a_0 = \gamma(x)$ and $a_r \ne 0$. 




\subsection{Elliptic modules}
\label{drinfeld_section}

An \textit{elliptic module} is a rank-2 Drinfeld module, obtained by setting $r = 2$ in 
\eqref{equ:Drinfeld}. In this paper, we consider elliptic modules over $\F_q(x)$, i.e. $L = 
\F_q(x)$ is the field of fractions of $A$. In particular, $\phi$ will always be given as the ring 
homomorphisms
\[
\begin{array}{rrll}
	\phi : & \F_q[x] & \longrightarrow & \F_q(x)\ang{\tau} \\
	& x & \longmapsto & x + g_\phi \tau + \Delta_\phi \tau^2	
\end{array}
\]
for some $g_\phi \in \F_q[x]$ and nonzero $\Delta_\phi \in \F_q[x]$. Unless otherwise noted, a 
Drinfeld module over $\F_q(x)$ will mean an elliptic module as above. 

For an irreducible polynomial $f \in \F_q[x]$, if $\Delta_\phi$ is nonzero modulo $f$, then the 
reduction $\phi/f$ of $\phi$ at $f$ is defined as the ring homomorphism
\[
\begin{array}{rrll}
	\phi/f : & \F_q[x] & \longrightarrow & (\F_q[x]/f) \ang{\tau} \\
	& x & \longmapsto & x + (g_\phi \bmod f) \tau + (\Delta_\phi\bmod f) \tau^.
\end{array}
\]
The image of $a \in \F_q[x]$ under $\phi/f$ is denoted by $(\phi/f)(a)$. Even if $\Delta_{\phi}$ is 
zero modulo $f$, one could still obtain the reduction $(\phi/f)$ of $\phi$ at $f$ through
minimal models of $\phi$; see \cite{gek1}. We refrain from addressing this case since it will not 
be required. For a non zero ideal $\f \subset \F_q[x]$, let $\deg(\f)$ denote the degree of its 
monic generator.


\subsection{Hasse invariants and Deligne's congruence}

Let $\phi$ be an elliptic module as above and let $f \in \F_q[x]$ be an irreducible monic 
polynomial not dividing $\Delta_\phi$. The \textit{Hasse invariant} $h_{\phi,f} \in \F_q[x]/f$ of 
$\phi$ at $f$ is the coefficient of $\tau^{\deg(f)}$ in the expansion 
\[ \ph_f = \sum_{i=0}^{2\deg(f)} h_i \tau^i \in (\F_{q}[x]/f)\{\tau\}. \]
The elliptic module $\phi$ has supersingular reduction at $f$ if and only if $h_{\phi,f}$ vanishes 
\cite{gos}. Otherwise, $\phi$ is said to have ordinary reduction at $f$.

Recursively define a sequence $(r_{\phi,k})_{k \in \mathbb{N}}$ in $\F_q[x]^\mathbb{N}$ as 
$r_{\phi,0}:=1$, $r_{\phi,1}:=g_\phi$ and for $m>1$,
\begin{equation}
\label{eisenstein_recurrence}
	r_{\phi,m} := g_\phi^{q^{m-1}}r_{\phi,m-1} - (x^{q^{m-1}}-x)\D_\phi^{q^{m-2}} r_{\phi,m-2} \in 
	\F_q[x].
\end{equation}
Gekeler \cite[Eq 3.6, Prop 3.7]{gek} showed that $r_{\phi,m}$ is the value of the normalized 
Eisenstein series of weight $q^{m}-1$ on $\phi$ and established Deligne's congruence for elliptic 
modules. That is, for any $f$ of degree $k \geq 1$ with $\Delta_\phi \neq 0 \bmod f$, we have
\begin{equation}
\label{deligne_congruence}
	h_{\phi,f}(x) = r_{\phi,k} \pmod{f}.
\end{equation}
Hence, $r_{\phi,k}$ is in a sense a lift to $\F_q[x]$ of all the Hasse invariants of $\phi$ at 
primes of degree $k$. If $\phi$ has supersingular reduction at an irreducible polynomial $f$ of 
degree $k$, that is, $h_{\phi,f} = 0$, then by Deligne's congruence we have $r_{\phi,k} = 0 \bmod 
f$. From the recurrence \eqref{eisenstein_recurrence}, it follows that $r_{\phi,k+1} = 0 \bmod f$, 
since $f$ divides $x^{q^k} - x$. Plugging $r_{\phi,k} = r_{\phi,k+1} = 0 \bmod f$  into the 
recurrence \eqref{eisenstein_recurrence} yields
\begin{equation}
\label{supersingular_zero}
	r_{\phi,j} = 0 \pmod{f}, ~ \forall~j \ge k.
\end{equation}
On the other hand, if $\phi$ has ordinary reduction at $f$, then \cite[Lemma~2.3]{cor}
\begin{equation}
\label{supersingular_nonzero}
	r_{\phi,j} \neq 0 \pmod{f}, ~ \forall~j \ge k.
\end{equation}
This suggests that we could use an elliptic module $\phi$ in a polynomial factorization algorithm 
by separating supersingular primes from ordinary ones. For most elliptic modules, the density of 
supersingular primes is too small for this to work. However, for a special class, namely elliptic 
modules with complex multiplication, the density of supersingular primes is $1/2$. 




\subsection{Elliptic modules with complex multiplication}
\label{randomized_section}

An elliptic module $\phi$ is said to have complex multiplication by an imaginary quadratic 
extension $L/\F_q(x)$ if $\End_{\F_q(x)}(\phi)\otimes_{\F_q[x]} \F_q(x) \cong L$. Here, $L$ is 
imaginary if the prime $(1/x) \in \F_q(x)$ at infinity does not split in $L$. For a $\phi$ with 
complex multiplication by $L$, a prime $\p$ that is unramified in $L$ is supersingular if and
only if $\p$ is inert in $L$.

This suggests the following strategy to factor a monic square free polynomial $f(x) \in \F_q[x]$. 
Suppose $f(x)$ factors into monic irreducible polynomials as $f(x) = \prod_i p_i(x)$. Pick an 
elliptic module $\phi$ with complex multiplication by some imaginary quadratic extension 
$L/\F_q(x)$. Compute $r_{\phi,k}(x) \bmod f$ for some $k \le \deg(f)$. By equations 
\eqref{supersingular_zero} and \eqref{supersingular_nonzero}, 
\[\gcd(r_{\phi,k} \bmod f, f) = \prod_{\substack{(p_i) \text{ inert in } L \\ \deg(p_i) \le k}} 
p_i \]
is a factor of $f$. Since for every degree, roughly half the primes of that degree are inert in 
$L$, the factorization thus obtained is likely to be non trivial.


 
%####################################################
 
\section{Efficient computation of the Hasse invariant lift}
\label{sec:hasse}

In this section, we devise an efficient algorithm to compute the Hasse invariant lift. We let $\K = 
\F_q[x]/f(x)$ and define $\xi$ as the image of $x$ in $\K$. The recursion  
\ref{eisenstein_recurrence} for computing $r_{\phi,n}(x)$ can be written as
\[
\begin{bmatrix}
r_{\phi,k - 1} \\
r_{\phi,k} \\
\end{bmatrix} = 
\begin{bmatrix}
0 & 1 \\
-[k - 1]\Delta^{q^{k - 2}} & g^{q^{k - 1}}
\end{bmatrix}
\begin{bmatrix}
r_{\phi,k - 2} \\
r_{\phi,k - 1} \\
\end{bmatrix}.
\]
where $[k - 1]:=\xi^{q^{k - 1}}-\xi$. Define the following sequence of matrices in 
$\mathscr{M}_2(\K)$:
\[
A_k :=\begin{bmatrix}
0 & 1 \\
-[k - 1]\Delta^{q^{k - 2}} & g^{q^{k - 1}}
\end{bmatrix}.
\]
Then we have
\[
\begin{bmatrix}
r_{\phi,k - 1} \\
r_{\phi,k} \\
\end{bmatrix} = 
A_kA_{k - 1} \cdots A_2
\begin{bmatrix}
r_{\phi,0} \\
r_{\phi,1} \\
\end{bmatrix}.
\]
Our goal is to compute the product 
\[B_n := A_nA_{n - 1} \cdots A_2 ~ \in ~ \mathscr{M}(\K)\]
for then we can read off $r_{\phi,n}$ from $B_n\begin{bmatrix}
r_{\phi,0} \\
r_{\phi,1} \\
\end{bmatrix}.$ 
%for a given positive integer $m$. Although the following algorithm works for any $m$, we set $m = 
%n$ for the sake of simplicity, specially when analyzing the runtime complexity. 
%Define the linear map $\tau: \K \to \K$ given by $\tau(f) = f^q$ for any $f$ in $\K$. We can 
%extend the action of $\tau$ to the polynomial ring $\mathscr{M}_2(\K)[Y]$ by leaving $Y$ fixed and 
%acting on the coefficient matrices entry-wise.
Extend the $\F_q$-linear $q^{th}$-power Frobenius map $\tau: \K \to \K$ to the polynomial ring 
$\mathscr{M}_2(\K)[Y]$ by leaving $Y$ fixed and acting on the coefficient matrices entry-wise.
Let
\[
\mathcal{A} := 
\begin{bmatrix}
0 & 1 \\
-\tau(\xi)\Delta & \tau(g)
\end{bmatrix}
+
\begin{bmatrix}
0 & 0 \\
\Delta & 0
\end{bmatrix} Y ~ \in ~ \mathscr{M}_2(\K)[Y].
\]
For a $\mathcal{M} \in \mathscr{M}_2(\K)[Y]$ and a $\zeta \in \K$, let $\mathcal{M}(\zeta)$ denote 
the image of $\mathcal{M}$ under the substitution 
\[Y \longmapsto 
\begin{bmatrix}
\zeta & 0 \\
0 & \zeta
\end{bmatrix}.
\]
Then, for any $k \ge 1$, we have $$A_k = \tau^{k - 2}(\mathcal{A})(\xi).$$
Let $\ell := \lceil n^\beta \rceil$, $m := \lfloor n / \ell \rfloor$ and define 
\[\mathcal{B} := \tau^{\ell-1}(\mathcal{A}) \cdots \tau(\mathcal{A}) \mathcal{A}.\]
It follows from the above that 
\[\mathcal{B}(\xi) = A_{\ell+1}A_{\ell - 2} \cdots A_2.\]
More generally, using the fact that for all $i, j$
\[A_{i + j + 2} = \tau^{i + j}(\mathcal{A})(\xi) = \tau^j\Big(\tau^i(\mathcal{A})\big( 
\tau^{-j}(\xi)\big)\Big),\]
we deduce for all $i \ge 1$ that 
\[\tau^{i}\Big(\mathcal{B} \big( \tau^{-i}(\xi)\big) \Big) = A_{i + \ell+1} \cdots A_{i + 3}A_{i + 
	2}.\]
In particular, $B_n$ can be computed as the product of the following matrices, 
\[
\mathcal{B} \big(\xi\big), \tau^{\ell}\Big(\mathcal{B} \big( \tau^{-\ell}(\xi)\big) \Big), \dots, 
\tau^{m \ell}\Big(\mathcal{B} \big( \tau^{-m \ell}(\xi)\big) \Big).
\]
This suggest Algorithm \ref{alg:hasse-inv} for computing $B_n$.

\begin{algorithm}[H]
	\caption{Compute Hasse invariant}
	\label{alg:hasse-inv}
	\begin{algorithmic}[1]
		\REQUIRE A degree $n$ polynomial $f(x)$ and the coefficients $\Delta(x), g(x)$ (modulo 
		$f(x)$) of a Drinfeld module $\phi$.
		\ENSURE The $n$-th Hasse invariant lift $r_{\phi,n}(x)$ modulo $f(x)$.
		\STATE Let $\ell := \lceil n^\beta \rceil$, and let $m := \lceil n / \ell \rceil$
		\STATE\label{step:hasse-2}
		Compute $\mathcal{B} = \tau^{\ell-1}(\mathcal{A}) \cdots \tau(\mathcal{A}) \mathcal{A}$
		\STATE\label{step:hasse-3}
		Compute $\xi_i = \tau^{-i\ell}(\xi)$ for $0 \le i \le m$
		\STATE\label{step:hasse-4}
		Compute $\beta_i = \mathcal{B}(\xi_i)$ for $0 \le i \le m$
		\STATE\label{step:hasse-5}
		Compute $t_i = \tau^{i\ell}(\beta_i)$ for $0 \le i \le m$
		\RETURN\label{step:hasse-6} 
		$r_{\phi,n}$, by reading off from $B_n$ computed as the product $t_m t_{m - 1} \cdots t_0$.
	\end{algorithmic}
\end{algorithm}

The correctness of the algorithm follows from the preceding remarks. The next two subsections 
analyze its complexity in both boolean and algebraic models.


\subsection{Complexity in an algebraic model}

First we need a lemma from \cite{ks} for efficient simultaneous modular composition, which we 
include here for convenience:

\begin{lemma}[\cite{ks}]
	\label{lemma:ks}
	Given a polynomials $f \in \K[x]$ of degree $n$ over an arbitrary field $\K$, and polynomials 
	$g_1, \dots, g_k, h \in \K[x]$ of degree less than $n$, where $k = O(n)$, we can compute 
	\[ g_1(h) \bmod f, \dots, g_k(h) \bmod f \]
	using $O(n^{(\omega + 1) / 2} k^{(\omega - 1) / 2})$ operations in $\K$.
\end{lemma}

In this subsection we prove the following:

\begin{theorem}
	\label{theo:hasse-inv}
	Algorithm \ref{alg:hasse-inv} runs in
	\[O(n^{(\omega + 1) / 2 + (1 - \beta)(\omega - 1) / 2} + n^{1 + \beta + o(1)}\log q)\] 
	operations in $\F_q$, where $\omega$ is the matrix multiplication exponent. Taking $\omega 
	\approx	2.375$ and $\beta \approx 0.815$, the above is $O(n^{1.815}\log q)$ operations in 
	$\F_q$.
\end{theorem}

\begin{proof}
	Step \ref{step:hasse-2} is done using $\ell$ successive applications of the Frobenius and 
	multiplying the results. Here we do the Frobenius using binary-powering. The cost is 
	$O(\ell\MM(n)\log q + \MM(\ell n)) = O(n^{1 + \beta + o(1)}\log q)$ operations in $\F_q$.
	
	Step \ref{step:hasse-3} is computed as follows. First we compute $g = \tau^{m\ell}(\xi)$ by 
	computing $\tau(\xi)$ using binary-powering and then doing $\log (m\ell)$ modular polynomial 
	compositions. This takes $O(\CC(n)\log n + \MM(n)\log q)$ operations in $\F_q$. Now we find $f 
	\in 
	\K$ such that \[f^{q^{m\ell}} = \tau^{m\ell}(f) = f(g) = \xi.\]
	In other words we find $f = \tau^{-m\ell}(\xi)$. This can be done using transposed modular 
	composition with cost dominated by previous steps, see \cite{DeDoSc2014} for more details. So 
	we 
	have $\xi_m = f$, and other $\xi_i$ for $i < m$ can be computed using Lemma \ref{lemma:ks}. 
	This 
	takes $O(n^{(\omega + 1) / 2}n^{(1 - \beta)(\omega - 1) / 2}) = O(n^{(\omega + 1) / 2 + (1 - 
		\beta)(\omega - 1) / 2})$ operations in $\F_q$.
	
	Step \ref{step:hasse-4} can be done using multipoint evaluation \cite{vzGG}. We are evaluating 
	a 
	polynomial of degree at most $\ell$ at $m$ points. This takes $O((\ell / m + \log m) 
	\MM(m\ell)) = 
	O(n^{2\beta + o(1)})$ operations in $\F_q$.
	
	Step \ref{step:hasse-5} is done as follows. Assume we have the polynomials $h_i = 
	\tau^{i\ell}(\xi)$, $1 \le i \le m$ from the previous steps. One can compute these polynomials, 
	if
	necessary, at the cost of $O(n^{(\omega + 1) / 2 + (1 - \beta)(\omega - 1) / 2})$ using Lemma 
	\ref{lemma:ks}. Computing a value $\tau^{s\ell}(\beta_i)$ is equivalent to computing the 
	modular 
	composition $\beta_i(h_s)$. Consider the list $\beta_1, \beta_2, \dots, \beta_m$. One can 
	produce 
	all the powers $\tau^{\ell}(\beta_1), \dots, \tau^{m\ell}(\beta_m)$ only using composition with 
	$h_1, h_2, h_{2^2}, \dots, h_{2^d}$ where $d = \lfloor \log m \rfloor$. More precisely, all the 
	$\beta_i$ with the $k$-th bit of the binary representation of $i$ equal to '1' are composed 
	with 
	$h_{2^k}$. There are $m / 2$ of these $\beta_i$. Therefore, $O(\log m)$ applications of Lemma 
	\ref{lemma:ks} at the cost of $O(n^{(\omega + 1) / 2 + (1 - \beta)(\omega - 1) / 2})$ gives the 
	final result. 
	
	Step \ref{step:hasse-6} is done using $O(m\MM(n)) = O(n^{2 - \beta + o(1)})$ operations in 
	$\F_q$ 
	which is dominated by the rest.
\end{proof}


\subsection{Complexity in a boolean model}

In this subsection we set $\beta = 1/2$. Curiously, each of the steps \ref{step:hasse-2}, 
\ref{step:hasse-3}, \ref{step:hasse-4} and \ref{step:hasse-5} can each be performed in 
$n^{3/2+o(1)} (\log q)^{1+o(1)} + n^{1+o(1)} (\log q)^{2+o(1)}$. time. We next briefly sketch 
how.\\ 

Step \ref{step:hasse-2} is performed recursively. Given 
$\tau^{\lfloor\ell/2\rfloor}(\mathcal{A})\tau^{\lfloor\ell/2\rfloor-1}\ldots\tau(\mathcal{A}) 
\mathcal{A}$, it takes one $\tau^{\lfloor\ell/2\rfloor}$ map (and a constant number of 
multiplications) in $\mathscr{M}_2(\K)[Y]$ to compute $\tau^{\ell-1}(\mathcal{A}) \tau^{\ell-2} 
\ldots \tau(\mathcal{A}) \mathcal{A}$. The $\tau^{\lfloor\ell/2\rfloor}$ power map is computed 
using the iterated Frobenius algorithm of von zur Gathen and Shoup \cite{gs} implemented using fast 
modular composition \cite{ku}. Step \ref{step:hasse-3} is performed using Lemma \ref{lemma:ks} 
implemented using fast modular composition \cite{ku}. Step \ref{step:hasse-5} is nearly identical, 
except its performed over $\mathscr{M}_2(\K)$. Step \ref{step:hasse-4} is performed using 
multipoint evaluation \cite{vzGG}. In summary, we have proven 
\begin{theorem}
	\label{theo:hasse-inv}
	Algorithm \ref{alg:hasse-inv} runs in 
	\[O(n^{3/2+o(1)} (\log q)^{1+o(1)} + n^{1+o(1)} (\log q)^{2+o(1)})\]
	bit operations.
\end{theorem}



 
%####################################################
 

\section{Randomized Polynomial Factorization using Drinfeld Modules with Complex Multiplication}
\subsection{Constructing Drinfeld Modules with Complex 
Multiplication}\label{drinfeld_construction_subsection}

Our strategy is to pick an $a \in \F_q$ at random and construct a Drinfeld module $\phi$ with complex multiplication by the imaginary quadratic extension $\F_q(x)(\sqrt{d(x)})$ of discriminant $d(x):=x-a$. From \cite{dor}, the Drinfeld module $\phi^\prime$ with $$g_{\phi^\prime}(x):=\sqrt{d(x)}+\left(\sqrt{d(x)}\right)^q, \D_{\phi^\prime}(x) := 1$$ has complex multiplication by $\F_q(x)(\sqrt{d(x)})$.\\ \\
However, $\phi^\prime$ has the disadvantage of not being defined over $A$ since $g_{\phi^\prime}(x) \notin A$.\\ \\
We construct an alternate $\phi$, that is isomorphic to $\phi^\prime$ but defined over $A$. The $J$-invariant \cite{gek} of $\phi^\prime$ is $$J_{\phi^\prime}(x) := \frac{g_{\phi^\prime}(x)^{q+1}}{\D_{\phi^\prime}(x)} = d(x)^{\frac{q+1}{2}}\left(1+d(x)^{\frac{q-1}{2}}\right)^{q+1}.$$
With the knowledge that two Drinfeld modules with the same $J$-invariant are isomorphic, we construct the Drinfeld module $\phi$ satisfying $$g_\phi(x)^{q+1} = (J_{\phi^\prime}(x))^2, \D_{\phi}(x)= J_{\phi^\prime}(x)$$ thereby ensuring $$J_{\phi}(x)=J_{\phi^\prime}(x).$$ Further, this assures that $\phi$ is defined over $A$ since

$$g_\phi(x):=d(x)(1+d(x)^{\frac{q-1}{2}})^2, \quad 
\D_\phi(x):=d(x)^{\frac{q+1}{2}}(1+d(x)^{\frac{q-1}{2}})^{q+1}.$$
In summary, $\phi$ has complex multiplication by $\F_q(x)(\sqrt{d(x)})$) and is defined over $A$.
\subsection{Polynomial Factorization using Drinfeld Modules with Complex Multiplication}
We now state our randomized algorithm to factor polynomials over finite fields using Drinfeld modules with complex multiplication. Curiously, it can be stated and implemented with no reference to Drinfeld modules.


\begin{algorithm}[H]
	\caption{Polynomial factorization}
	\label{factoring_algorithm}
	\begin{algorithmic}[1]
		\REQUIRE Monic squarefree $f(x) \in \F_q[x]$ of degree $n$
		\ENSURE The irreducible factors of $f(x)$
		\STATE If $f(x)$ is irreducible then print $f(x)$ and return.
		\STATE If $f(x)$ has roots in $\F_q$, find and remove the roots.
		%\STATE Let $\K = \F_q[x] / f$
		\STATE Pick $a \in \F_q$ uniformly at random and compute \\ % the following elements in $\K$ \\
		$d(x) := x-a \mod f(x)$. \\
		$g_\phi(x) := d(x)(1+d(x)^{\frac{q-1}{2}})^2 \mod f(x)$. \\
		$\Delta_\phi(x) := d(x)^{\frac{q+1}{2}}(1+d(x)^{\frac{q-1}{2}})^{q+1} \mod f(x)$ .
		\STATE Compute $r_{\phi,n}(x) \mod f(x)$ as in recurrence (\ref{eisenstein_recurrence}) 
		(that is, the $n$-th Hasse invariant lift with respect to the Drinfeld module $\phi$ with 
		coefficients $g_\phi, \Delta_\phi$, computed modulo $f$)
		\STATE Compute $f_1(x) := \gcd(r_{\phi,n}(x), f(x))$ and recursively factor $f_1(x)$ and $f(x)/f_1(x)$.
	\end{algorithmic}
\end{algorithm}

\begin{remark}
	In Step $4$, we may compute $r_{\phi,k} \mod f(x)$ for some $k\geq n$ instead of $r_{\phi,k} \mod f(x)$. As is evident from the forthcoming paragraphs, the rest of the algorithm would run identically. For instance, it might be convenient in practice to chose $k$ to be the smallest power of $2$ that is at least $n$.
\end{remark}


The irreducibility test in Step $1$ can be performed in $O(\CC(n)\log n + \MM(n)\log q)$ operations 
in $\F_q$ \cite{vzGG}, or $O(n^{1+o(1)} (\log q)^{2+o(1)})$ bit operations. In Step $2$, all the 
linear factors of $f(x)$ are found and removed using a root finding algorithm. It takes 
$O(\MM(n)\log n \log(nq))$ operations in $\F_q$ \cite{vzGG}, or $O(n^{1+o(1)} (\log q)^{2+o(1)})$ 
bit operations. \\

 In Step $3$, we choose $a \in \F_q$ at random and construct a Drinfeld module $\phi$ with complex 
 multiplication by $\F_q(x)(\sqrt{x-a})$. The primes that divide $\D_\phi(x)$ are precisely 
 $\{(x-b),  b \in \F_q, \sqrt{b-a} \notin \F_q\} \cup \{(d(x))\}$. We might run into issues of bad 
 reduction if the polynomial $f(x)$ to be factored had roots. It is to prevent this, we performed 
 root finding in Step $2$. \\

 In Step $4$, from the recurrence \ref{eisenstein_recurrence}, we indeed compute the $n^{th}$ Hasse 
 invariant lift $r_{\phi,n}(x)$ modulo $f(x)$. By Deligne's congruence \ref{deligne_congruence}, a 
 degree $k$ monic irreducible factor $p(x)$ of $f(x)$ divides $r_k(x)$ if and only if $(p(x))$ is 
 supersingular with respect to $\phi$. In particular, $\gcd(r_{\phi,k}(x),f(x))$ is the product of 
 all degree $k$ irreducible factors of $f(x)$ that are supersingular with respect to $\phi$. 
 Further, by equations \ref{supersingular_zero} and \ref{supersingular_nonzero}, 
 $\gcd(r_{\phi,k}(x),f(x))$ is the product of all degree at most $k$ irreducible factors of $f(x)$ 
 that are supersingular with respect to $\phi$. In particular, $(f_1(x) =) 
 \gcd(r_{\phi,k}(x),f(x))$ is the product of irreducible factors of $f(x)$ that are supersingular 
 with respect to $\phi$. Thus our algorithm separates the irreducible factors supported at the 
 supersingular primes from those supported at the ordinary primes. All that remains to argue is 
 that for a Drinfeld module chosen as randomly in Algorithm \ref{factoring_algorithm}, this split 
 is random enough to ensure that the recursion depth is logarithmic in $n$.\\ 

%Hence at iteration $k$ in Step $4$, by Deligne's congruence \ref{deligne_congruence}, a degree $k$ monic irreducible factor $p(x)$ of $f(x)$ divides $r_k(x)$ if and only if $(p(x))$ is supersingular with respect to $\phi$. In particular, $\gcd(r_k(x),f(x))$ is the product of all degree $k$ irreducible factors of $f(x)$ that are supersingular with respect to $\phi$. If there is only one such factor, we output it. Else, the product is multiplied to $f_{ss}(x)$ to be split recursively later. Likewise, at iteration $k$ in Step $4$, $\gcd(x^{q^k}-x,f(x))/\gcd(r_{k}(x),f(x))$ is the product of all degree $k$ irreducible factors of $f(x)$ that are ordinary with respect to $\phi$. If there is only one such factor, we output it. Else, the product is multiplied to $f_{or}(x)$ to be split recursively later.\\ 

 The following Lemma \ref{splitting_lemma} states that any two distinct factors of $f(x)$ of the 
 same degree are neither both supersingular nor both ordinary with probability $1/2$. This ensures 
 that the splitting into supersingular and ordinary factors in Step $4$ is random enough that the 
 recursion depth of our algorithm is logarithmic in $n$. For the lemma to apply to Algorithm 
 \ref{factoring_algorithm}, we need to assume $\sqrt{q} \geq n$.  
The assumption $\sqrt{q} \geq  n$ in Algorithm \ref{factoring_algorithm} can thankfully be made without loss of generality in theory. For if $\sqrt{q} < n$, we might choose to factor over a slightly larger field $\F_{q^\prime}$ where $q^\prime$ is the smallest power of $q$ such that $\sqrt{q^\prime} > n$ and still recover the factorization over $\F_q$ (c.f. \cite[Remark 3.2]{nar}). Further, the running times are only affected by logarithmic factors. In practice, we recommend running Algorithm \ref{factoring_algorithm} as is and take Lemma \ref{splitting_lemma} as strong evidence for the recursion depth to be logarithmic in $n$.


\begin{lemma}\label{splitting_lemma}
Let $p_1(x),p_2(x) \in A $ be two distinct monic irreducible polynomials of degree $k$ where $1<k \leq \sqrt{q}$. Let $\phi$ be a Drinfeld module with complex multiplication by the imaginary quadratic extension $\F_q(x)(\sqrt{x-a})$ where $a \in \F_q$ is chosen at random. With probability close to $1/2$, exactly one of $(p_1(x))$ or $(p_2(x))$ is supersingular with respect to $\phi$.
\end{lemma}
\begin{proof}
Since $k > 1$ neither $(p_1(x))$ nor $(p_2(x))$ ramify in $\F_q(x)(\sqrt{x-a})$. Hence, the probability that exactly one of $(p_1(x))$ or $(p_2(x))$ is supersingular with respect to $\phi$ is precisely the probability that exactly one of $(p_1),(p_2)$ splits in $\F_q(x)(\sqrt{x-a})/\F_q(x)$.\\ \\
%That is, exactly one of $(p_1),(p_2)$ splits in $\F_q(t)(\sqrt{d})/\F_q(t)$.
For $i\in\{0,1\}$, let $K_i:=\F_q(x)(\alpha_i)$ be the hyperelliptic extension of $\F_q(t)$ 
obtained by adjoining a root $\alpha_i$ of $y^2-p_i(x)$. By quadratic reciprocity over function 
fields \cite{carlitz1932}, since $p_1(x)$ and $p_2(x)$ have the same degree, exactly one of 
$(p_1(x)),(p_2(x))$ splits in $\F_q(x)(\sqrt{x-a})$ if and only if $x-a$ is split in exactly one of 
$K_1,K_2$. That is, $(x-a)$ is neither completely split nor completely inert in the composite 
$K_1K_2$. Since $p_1(x)$ and $p_2(x)$ are distinct, $K_1$ and $K_2$ are linearly disjoint over 
$\F_q(x)$. Further, $K_1K_2$ is Galois over $\F_q(t)$ with $$Gal(K_1K_2/\F_q(x)) \cong 
Gal(K_1/\F_q(x)) \times Gal(K_2/\F_q(x)) \cong \Z/2\Z \oplus \Z/2\Z.$$  For $(x-a)$ to be neither 
totally split nor totally inert, the Artin symbol $$((x-a),K_1K_2/\F_q(x)) \in 
Gal(K_1K_2/\F_q(x))$$ has to be either $(0,1)$ or $(1,0)$ under the isomorphism 
$Gal(K_1K_2/\F_q(x)) \cong \Z/2\Z \oplus \Z/2\Z$. Applying Chebotarev's density theorem, the number 
$N$ of degree one primes $\{(x-a),a\in \F_q\}$ that are neither totally inert nor totally split in 
$K_1K_2$ is bounded by 
$$ \left|N - \frac{q}{2} \right| \leq 2 g(K_1K_2)\sqrt{q}$$
where $g(K_1K_2)$ is the genus of $K_1K_2$. By the Riemann-Hurwitz genus formula, $g(K_1K_2) = k-1 \leq \sqrt{q}/2$. Hence when $a \in \F_q$ is chosen at random, $(x-a)$ is neither totally inert nor totally split in $K_1K_2$ with probability close to $1/2$. 
\end{proof}
In summary, we have thus proven that Algorithm \ref{factoring_algorithm} is correct and that it has 
expected recursion depth logarithmic in $n$. Further, the bottleneck is Step $4$, the computation 
of the Hasse invariant lift. A naive computation relying on recurrence \ref{eisenstein_recurrence} 
would lead to a runtime dependence quadratic in $n$. In the forthcoming section we describe a 
deterministic algorithm Algorithm \ref{alg:hasse-inv} to compute the Hasse invariant lift with 
runtime subquadratic in $n$. Incorporating Algorithm \ref{alg:hasse-inv} in Algorithm 
\ref{factoring_algorithm} thus leads to Theorem \ref{theorem:main}.






\bibliographystyle{plain}
\bibliography{references}
\end{document}



